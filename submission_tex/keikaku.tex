\documentclass[uplatex]{jsarticle}
\usepackage[dvipdfmx]{graphicx, color}
\usepackage{ascmac} % 複数行の枠
\usepackage{nccmath} % 複数行(数式等)左揃え
\usepackage{amssymb} % 行列のでかい括弧など
\usepackage{amsmath} % align, equation
\usepackage{mathtools} % for :=, =:
\usepackage{bm} % vector
\usepackage{here} % tables [H]
\usepackage{longtable}

\title{アルゴリズムB\\計画書}
\author{1W183011-7 糸井琢人\and 1W172143-0 佐藤将輝\and 1W182164-3 城武秀祐\and 1W182171-3 助川裕太\and 1W182339-5 山田啓太\and 1W183129-6 柳田侑羽}
\date{\today}

\begin{document}
    \maketitle
    \section{4班}

    \section{メンバー}
    \begin{itemize}
        \item 1W183011-7 糸井琢人
        \item 1W172143-0 佐藤将輝
        \item 1W182164-3 城武秀祐
        \item 1W182171-3 助川裕太
        \item 1W182339-5 山田啓太
        \item 1W183129-6 柳田侑羽
    \end{itemize}

    \section{作業の目的}
    虫食いデータと刻んだデータが与えられたとき、できるだけ早く、そして精度よく元のデータを復元する。

    \section{作業計画}
    \begin{center}
        \begin{longtable}{l|l|l}
            Due Date&内容&補足\\\hline
            \endfirsthead
            %------ 2ページ以降の表の最上部 ----
            \multicolumn{2}{c}{前ページからの続き} \\ \hline
            Due Date&内容&補足\\\hline
            \endhead
            %----- ページの表の最下部 --------
            \hline
            \multicolumn{2}{c}{次ページに続く} \\
            \endfoot
            %----- 最終ページの表の最下部 --------
            \hline
            \multicolumn{2}{c}{以上} \\
            \endlastfoot
            11/29&Git垢作成&全員\\
            &Gitレポジトリ作成&城武\\
            &コラボレーターに追加&城武\\
            &環境構築&全員\\\hline
            12/6&入力実装&佐藤\\
            &ヘッダファイル&城武(随時更新)\\
            &Queue実装&糸井\\
            &Aho-corasick実装&糸井\\
            &素晴らしいアルゴリズムを考える&全員\\\hline
            12/13&考案したものの実装・テスト&地獄\\\hline
            12/22&中間計測&\\\hline
            1/10&圧縮ファイル提出&\\\hline
            1/13&発表資料提出&\\\hline
            1/24&最終提出&\\\hline
        \end{longtable}
    \end{center}

    \section{メンバーの役割分担}
    \begin{itemize}
        \item 入力実装:佐藤
        \item アルゴリズム考える人:全員
        \item ヘッダファイル作る人:城武
        \item データ構造の実装:糸井、城武
        \item コード実装班:全員
        \item 発表する人(原稿):糸井
        \item 発表ポスター:糸井
        \item コード直す人:城武
        \item 城武のコード直す人:糸井
        \item 随時追加…
    \end{itemize}
\end{document}